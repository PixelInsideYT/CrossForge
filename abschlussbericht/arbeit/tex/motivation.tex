% vim: set ts=4 sw=4 smartindent expandtab textwidth=100:

\section{Projektziel Innenstadt}

Gemeinsam als Gruppe haben wir uns für das Szenario Innenstadt entschieden. Inspiriert von Solarpunk ist es das Ziel ein modernes, menschenorientiertes aber trotzdem platzeffizientes Gelände zu erschaffen. Der Baustil soll modern und luxuriös aber trotzdem mit der Natur verbunden sein. Deswegen möchten wir die Umgebung mit geschwungenen organischen Formen gestalten und verschiedene moderne und klassische Materialien wie Granit, Sandstein, Holz und Glas mit einander verbinden. Die Gebäude sollen über Brücken miteinander verbunden werden, sodass man nicht ins Erdgeschoss zurück gehen muss, um von einem Haus zum nächsten zu gelangen. An allen Fassaden und in den öffentlichen Aufenhaltsbereichen wachsen Planzen, um für ein angenehmes Klima zu sorgen.

In dieser Stadt leben Menschen und werden von verschiedenen Robotern im Alltag unterstützt. Die Roboter übernehmen dabei die Arbeit, die nicht erfüllend, stupide oder gefährlich ist. Putzroboter halten Fußböden sauber, entleeren Mülleimer und heben Müll auf. Sie können auch komplexere Oberflächen wie Toiletten, Waschbecken oder Türgriffe reinigen. Essensverkäufer-Roboter arbeiten in Geschäften wie McDonalds, Nordsee oder Subway und nehmen Bestellungen entgegen und bereiten diese zu. Gepäckroboter helfen den Menschen ihre Einkäufe nach Hause zu tragen. Gärtnerroboter gießen Pflanzen, sägen Äste ab und sperren Bereiche ab. Sie arbeiten mit Menschen zusammen, welche Anweisungen geben, welche Äste abgeschnitten werden sollen. 

Die Menschen sollen sich möglichst realistisch verhalten und haben ein eigenes Leben mit eigenen Zielen. Die Tagesabläufe von Kinder, Erwachsenen und Rentnern sollen sich je nach den individuellen Bedürfnissen unterscheiden. Diese Menschen interagieren mit den Robotern und arbeiten eng mit ihnen zusammen. Einige Kombinationen könnten zum Beispiel: Koch und Kellnerroboter, Gärtner und Gärtnerroboter oder Händler und Lagerroboter sein. Die Menschen interagieren aber nicht nur mit Robotern, sondern auch miteinander. Sie gehen gemeinsam Essen, verreisen miteinander und haben sinnvolle, natürliche Unterhaltungen.

Die Menschen haben auch tragbare Computergestützte Systeme, die sie im Alltag begleiten. So kann zum Beispiel eine AR-Brille die Navigation übernehmen und über ein HUD die Wegpunkte direkt in der Welt anzeigen. Zusätzlich können noch weitere Informationen wie Ankünfte, Wartezeiten oder die nächsten Termine dargestellt werden.

\section{Minimum Viable Product}

Da das Projektziel zu umfangreich für uns ist, haben wir es stark vereinfacht. Wir haben darauf geachtet, dass die Grundidee identisch bleibt und dass das originale Projektziel aus dem MVP entstehen kann. Die Szene soll aus zwei Platformen bestehen, die über eine Rampe miteinander verbunden sind. Auf jeder Platform befinden sich Pflanzen, die von fahrenden Robotern gegossen werden, wenn sie Wasser benötigen. Als Spieler kann man die Roboter fragen, um welche Pflanze es sich handelt.