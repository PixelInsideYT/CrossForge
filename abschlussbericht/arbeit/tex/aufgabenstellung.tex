% vim: set ts=4 sw=4 smartindent expandtab textwidth=100:

Im Sonderforschungsbereich Hybrid Societies wird die Interaktion von Menschen und
verkörperten digitalen Technologien (EDTs) untersucht. EDTs können autonome Fahrzeuge oder Roboter, Service- oder Info-Roboter, virtuelle Charaktere oder Menschen sein,
die mit technischen Erweiterungen wie intelligenten Brillen, Prothesen oder Exoskeletten
ausgestattet sind. Obwohl uns im täglichen Leben noch wenige dieser Dinge begegnen,
ergeben sich viele technische, logistische und soziale Implikationen, welche sich mit dem
Einzug solcher Technologien in unseren Alltag ergeben und bereits jetzt untersucht werden
sollten. Diese Implikationen betreffen unter anderem die Wahrnehmung, Kommunikation,
Interaktion, Umsetzung und den räumlichen Bedarf der jeweiligen Akteure. Die Erforschung dieser Aspekte ist schwierig, da die entsprechenden Geräte und Roboter entweder
noch nicht existieren oder sehr teuer sind. Daher ist es praktisch, eine virtuelle Umgebung
zu haben, in der diese Aspekte erforscht werden können.


In diesem teamorientierten Praktikum werden Möglichkeiten erforscht und umgesetzt,
wie eine lebendige, interaktive Welt erschaffen werden kann, in der ein Nutzer mit EDTs
interagieren kann oder die Interaktion von Agenten untereinander und in der Welt beobachten kann. Um dies zu erreichen, müssen die entsprechenden Objekte wie die Welt, die
Agenten und die Charaktere erschaffen werden. Zusätzlich müssen die Objekte interagierbar gemacht werden, und die Agenten müssen sich selbständig in der Welt bewegen und
mit ihr interagieren können. Der Fokus sollte dabei auf Glaubwürdigkeit und Plausibilität
liegen. Das bedeutet, dass Ansätze aus der wissenschaftlichen Literatur und bereits existierende EDTs verwendet werden sollten. Die virtuelle Welt sollte so angelegt sein, dass neue
Agenten in die Welt eingefügt oder das Verhalten bestehender Agenten geändert werden
kann, um die Auswirkungen zu untersuchen.