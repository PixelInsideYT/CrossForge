Unser Praktikumsteam bestand aus sechs Personen mit unterschiedlichen Wissensständen, weswegen ich die Aufgaben generell nach Schwierigkeit und Fähigkeiten der Bearbeiter verteilt habe. Angelique bricht ihr Informatikstudium ab und bekommt deshalb die kreative Aufgabe, die Modelle und die Szene zu erstellen. Lisa und Sophie haben noch sehr wenig Erfahrung und sollten deshalb die einfachsten Themen angehen. Auch Leon verfügt noch über wenig Erfahrung, aber dafür über eine gewisse Leidensfähigkeit. Mittelschwere bis schwere Aufgaben hat er ohne Zögern in Angriff genommen. Carlo und ich haben die meiste Erfahrung und arbeiten auch schon mehrere Stunden in der Woche als Werkstudenten. Deswegen haben wir uns auf die schwersten Probleme fokussiert.

Als Team haben uns darauf geeinigt, etwa sechs Stunden in der Woche dem Teamorientierten Praktikum zu widmen, da mehr neben Studium und Arbeit nicht möglich war. Deswegen haben wir uns auch gegen einen Scrum Prozess entschieden, weil die Retros, Plannings und Sprintwechsel zu viel Zeit in Anspruch nehmen. Um uns zu organisieren, wollten wir Jira einsetzen, jedoch hat die Software niemand wirklich benutzt. Da der Fortschritt langsam war, konnte man auch ohne Hilfsmittel sehr gut den Überblick darüber behalten, wer woran gearbeitet hat. Wir haben uns einmal in der Woche entweder über Discord oder in Präsenz getroffen, um uns zum Projektfortschritt zu synchronisieren.

\section{Aufgabenverteilung und Arbeitsweise}

Initial habe ich die Aufgaben wie folgt verteilt: Lisa implementiert das Dialogsystem, Sophie den Szeneneditor und ich das Multiagentensystem. Leon sucht eine ECS-Bibliothek und integriert diese, und Carlo beschäftigt sich mit dem Navigationssystem.

Da die meisten mit ihrer Aufgabe überfordert waren und nicht wussten, wo sie anfangen sollten, habe ich mit Lisa, Sophie und Leon jeweils eine Stunde in der Woche eine Pair-Coding-Session gehalten. Das hatte den großen Vorteil, dass ich dann genau wusste, wer welchen Fortschritt hat. Dem gegenüber standen zwei große Nachteile. Erstens konnte ich somit keine Features implementieren, weil meine sechs Stunden voll mit Pair Coding und Management waren, und zweitens ist es sehr verlockend, bei jeder kleinen Hürde aufzugeben und auf das nächste Pair Coding zu warten, weil das jeweilige Problem dort ja sowieso gelöst wird.
Zu diesem Zeitpunkt war Leon mit der Integration von Flecs und ich mit dem Konzept des Multiagentensystems fertig, so dass ich entschied, dass Leon die Implementierung von letzterem übernehmen kann.

In den Pair Coding Sessions haben wir gute Fortschritte mit dem Dialog- und Multiagentensystem gemacht. Die Aufgabe mit dem Szeneneditor bereitete Sophie keine Freude, und sie hat auch nicht verstanden, welche Schritte nötig sind und warum. Deswegen habe ich eigentlich nur diktiert, was sie schreiben soll. Das hat uns beiden jedoch keinen Spaß bereitet, weswegen wir zu dritt entschieden haben, dass es besser ist, wenn ich den Szeneneditor alleine fertigstelle. Sophie hat als nächstes Lisa mit beim Dialogsystem unterstützt.

In den Sommerferien haben wir die Pair Coding Sessions aufgrund von Prüfungen und Urlauben pausiert. Nach den Sommerferien habe ich eine Retrospektive vorbereitet und mit dem gesamten Team durchgeführt. Dabei haben sich verschiedene bevorzugte Arbeitsweisen heraus kristallisiert. Während Leon die Pair Coding Sessions bevorzugte und weiter führen wollte, sagten Lisa und Sophie, dass sie ihnen wenig bringen und sie eigenständig weiter arbeiten möchten. Auf diese Art und Weise haben wir dann auch bis zum Ende weiter gearbeitet.

\section{Analyse der Probleme}

Das Dialogsystem, an dem Lisa und Sophie gearbeitet haben, wurde nicht rechtzeitig fertig, weil es noch in das ECS integriert werden musste und das eine sehr zeitaufwendige Aufgabe ist. Generell habe ich darauf geachtet, dass sich die einzelnen Feature Branches nicht zu weit voneinander entfernen. Deswegen habe ich am 09.10.2023 alle Branches in den Master gemerged und danach den Master wieder in alle Featurebranches. Damit war der komplette Projektstand synchronisiert, und die beiden hätten anfangen können, das Dialogsystem in das Entity Component System zu integrieren. Wir haben aber gemeinsam entschieden, dass wir noch Zeit haben und das Dialogsystem zuerst fertig gestellt werden soll. Die beiden haben dann am 05.12.2023 begonnen, das Dialogsystem in das Entity Component System zu integrieren, sind jedoch nicht fertig geworden. Rückblickend wäre es sinnvoller gewesen, zuerst die Integration zu machen und danach das Dialogsystem fertig zu stellen.

Carlo hat das Praktikum, aus persönlichen Gründen, vor der Bewertung abgebrochen. Er besaß von uns allen die meiste C++ Erfahrung und arbeitet 18 Stunden in der Woche als Werkstudent, weswegen er die Aufgabe mit der größten Unsicherheit hinsichtlich Komplexität lösen sollte. Seine Aufgabe bestand darin, das Pathfinding System in CrossForge zu integrieren. Aufgrund von Studium, Arbeit, Umzug, Problemen mit seiner Entwicklungsumgebung und mäßiger Dokumentation von Detour hat er nur sehr langsam Fortschritte gemacht.  Bei der Retrospektive haben wir über diese Probleme gesprochen, und ich habe ihm vorgeschlagen eine andere, leichtere Aufgabe zu übernehmen. Er wollte nochmal einen Anlauf wagen, hat aber dann nach zwei Wochen mein Angebot angenommen. Ich habe seine Aufgabe übernommen, und er sollte als nächstes das Partikelsystem implementieren. Durch seine kontinuierlichen IDE Probleme hat er aber die Lust am Programmieren verloren. Das Partikelsystem hat dann Leon implementiert. Carlo sollte ab 12.12.2023 Angelique bei den Modellen und einer hübscheren Szene helfen. Er hat in der Woche vor Weihnachten noch die bestehende Szene ein wenig verbessert. Dann waren Weihnachtsferien und im Januar haben wir uns intensiv dem Bericht-Schreiben gewidmet.

Projektmanagement ist als Student schwer, weil man auf die intrinsische Motivation der Teammitglieder hoffen muss. Manager in der Industrie haben richtige Druckmittel (Mitarbeiter feuern) oder extrinsische Motivation (Lohn). Beides kann ich als studentischer Leiter nicht einsetzen. 
Zudem ist die Arbeitsverteilung, wie man in der Tabelle \ref{arbeitsaufteilung} gut erkennen kann, nicht gleichmäßig ausgefallen. Das ist auf die Tatsache zurückzuführen, dass die Teammitglieder unterschiedlich gut mit den ihnen zugeteilten Aufgaben zurechtgekommen sind. Mitglieder, die ihre Aufgaben schnell fertig gestellt haben, haben weitere übernommen wodurch sich das Ungleichgewicht weiter verstärkt hat.

\begin{figure}[h!]
\begin{tabular}{|c|c|c|c|c|c|c|}
\hline
Metric        & Leon         & Sophie    & Carlo     & Lisa      & Linus         & Angelique \\
\hline
Insertions    & 1.877 (7\%)  & 341 (1\%) & 296 (1\%) & 140 (0\%) & 23.092 (82\%) & 2.305 (8\%) \\
\hline
Deletions     & 1.010 (20\%) & 143 (3\%) & 6 (0\%)   & 43 (1\%)  & 3.775 (76\%)  & 3 (0\%) \\
\hline
Files         & 141 (35\%)   & 16 (4\%)  & 2 (0\%)   & 6 (1\%)   & 230 (57\%)    & 7 (2\%) \\
\hline
Commits       & 39 (33\%)    & 6 (5\%)   & 3 (3\%)   & 2 (2\%)   & 64 (55\%)     & 3 (3\%) \\
\hline
Lines changed & 2.887 (9\%)  & 484 (1\%) & 302 (1\%) & 183 (1\%) & 26.869 (81\%) & 2.308 (7\%) \\
\hline
\end{tabular}
\caption{Arbeitsverteilung anhand Git Historie im Zeitraum von 01.06.2023 bis 22.01.2024. Github Nutzernamen durch Klarnamen ersetzt}
\label{arbeitsaufteilung}
\end{figure}