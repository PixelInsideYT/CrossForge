% vim: set ts=4 sw=4 smartindent expandtab textwidth=100:
\documentclass[a4paper,
               final,
               draft,
%               oneside,    % einseitiges Layout für kleinere Arbeiten
               11pt]{book}

\usepackage{fontspec}   % eigene Schriftarten
\usepackage{microtype}  % besserer Textsatz
\usepackage{polyglossia}% Sprachbesonderheiten
\setdefaultlanguage[babelshorthands=true]{german}
\usepackage[german]{selnolig} % Ligaturen brechen, wo nötig

\usepackage{geometry}   % für Seiteneinrichtung
\usepackage[usegeometry,
            BCOR=10mm,
            twoside=semi]
           {typearea}   % automatische Berechnung der Seitenraender durch
                        %  dieses KOMA-Teilpacket, angeblich typographischen
                        %  Regeln folgend

\usepackage[printonlyused]{acronym}             % Das Packet, ein Abkuerzungsverzeichnis zu erstellen
    \renewcommand*{\acsfont}[1]{{\rmfamily #1}} % im Abkuerzungsverzeichnis auch roman verwenden
\usepackage[export]{adjustbox}
\usepackage[font=small,labelfont=bf]{caption}   % Bildunterschriften leicht anpassen
\usepackage{csquotes}   % landesspezifische Anführungszeichen
\usepackage{graphicx}   % Grafiken einbinden
\usepackage{icomma}     % Umgang mit Komma als Dezimaltrenner
\usepackage{ltablex}    % lange variabel breite Tabellen
\usepackage{newfile}    % um temporäre Dateien anzulegen
\usepackage{overpic}    % um auf Bildern TeX zu setzen
\usepackage[hyphens, spaces]{url}
\usepackage[backend=biber,
            maxbibnames=99,
            style=alphabetic,
            citestyle=alphabetic]
           {biblatex}   % für das Literaturverzeichnis

% scaling just of images, that are too wide
\makeatletter
\def\ScaleIfNeeded{%
    \ifdim\Gin@nat@width>\linewidth
        \linewidth
        \else
        \Gin@nat@width
        \fi
}
\makeatother
\setkeys{Gin}{width=\ScaleIfNeeded}

\parindent 2em          % Absatzerstzeileneinzuege (1em = Breite des M, 1ex = Hoehe des x
\parskip 0.5em          % Absatzabstaaende

\pagestyle{headings}

% Packages, die ueberall eingebunden werden muessen, z.B. damit Definitionen in dieser
% Datei funktionieren

%Default TU- und Fakultaetslogo
\newcommand{\tulogo}{pics/TU_Chemnitz_positiv_gruen}

% Hier den Namen des Verfassers eintragen, das ist eine Variable, weil sie eventuell oefters
% im Dokument vorkommt.
\newcommand{\dctitle}		{Abschlussbericht}
\newcommand{\dcauthortitle}	{Herr B.\,Sc.}
\newcommand{\dcauthorbirthdate}	{1. April 2000}
\newcommand{\dcauthorbirthplace}{Chemnitz}
\newcommand{\dcauthorlastname}	{Mustermann}
\newcommand{\dcauthorfirstname}	{Max}
\newcommand{\dcsubject}		{Masterarbeit}
\newcommand{\dcmatriculation}	{123456}

% ein kleines Hilfsmakro, um TODOs zu markieren
%\newcommand{\todo}[1]		{\hspace{0cm}\marginpar{\includegraphics[]{pics/todo.eps}}\textbf{-- TODO: #1 --}}
\newcommand{\todo}[1]		{}

% ein kleines Hilfmakro, mit dem ein Text am Anfang eines Gliederungspunktes
% eingefuegt werden soll, in dem ausgesagt wird, was gesagt werden soll.
% Fuer die finale Version wird dieses Makro einfach leer definiert.
%\newcommand{\whatsthis}[1]	{\hspace{0cm}\marginpar{\includegraphics[]{pics/what.eps}}{--- #1 ---}}
\newcommand{\whatsthis}[1]	{}


% setting up the pdf-a information
\newoutputstream{xmpstream}
\openoutputfile{\jobname.xmpdata}{xmpstream}
    \newcommand{\dummyspace}{ }

    \addtostream{xmpstream}{\protect\Title{\dctitle}}
    \addtostream{xmpstream}{\protect\Author{\dcauthorfirstname  \dcauthorlastname}}
    \addtostream{xmpstream}{\protect\Keywords{Compiler\protect\sep\dummyspace
                                              Database\protect\sep\dummyspace
                                              Index}}
    \addtostream{xmpstream}{\protect\Publisher{Technische Universität Chemnitz}}
\closeoutputstream{xmpstream}

% Das ist fuer die Metainformationen in der PDF-Datei und ist entsprechend anzupassen
%\usepackage{hyperref}
\usepackage[a-1b]{pdfx}

\hypersetup{unicode,
            pdftitle={\dctitle},
            pdfauthor={\dcauthorfirstname~\dcauthorlastname},
            pdfsubject={\dcsubject},
%           colorlinks=true,
%           citecolor=black,
%           filecolor=black,
%           linkcolor=black,
%           urlcolor=black,
            hidelinks}

% https://tex.stackexchange.com/questions/355071/pdfx-pdf-x-1a-standard-metadata-xmpdata-not-showing-in-pdf
\pdfinfo{
    /Title(\dctitle)
    /Author(\dcauthorfirstname\ \dcauthorlastname)
    /Subject(\dcsubject)
%   /Keywords(Compiler, Database, Index)
%   /PublicationType(Book)
}

\usepackage{makeidx}           % Das Packet, einen Stichwortindex zu erstellen
\makeindex                     % Indexerzeugung einschalten
\addbibresource{literatur.bib} % Bibliographiedatenbank laden

\setcounter{secnumdepth}{4}    % Numerierungstiefe für Überschriften
\setcounter{tocdepth}{4}       % Numerierungstiefe im TOC auf 3 setzen


\author{\dcauthorfirstname~\dcauthorlastname}
\title{\dctitle}
\date{\today}

\begin{document}
\frontmatter

\begin{titlepage}
    \sffamily
    \newgeometry{left=2cm,top=2cm,bottom=2cm,right=2cm}

    \centering{
        \begin{tabularx}{\linewidth}{lp{1.5em}|p{1.5em}X}
            \includegraphics[width=0.33\textwidth,valign=c]{\tulogo}
            & & &
            {\huge\bfseries Fakultät für Informatik}
        \end{tabularx}

        \vfill\vfill

        {\Huge\bfseries \dctitle}
        \vfill\vfill

        {\huge Teamorientiertes Praktikum}
        \vfill

        \vfill

        Angelique Gräfe\\[1ex]
        Lisa Neuhaus\\[1ex]
        Sophie Neuhaus\\[1ex]
        Carlo Kretzschmann\\[1ex]
        Leon Rollhagen\\[1ex]
        Linus Thriemer\\[1ex]
    \vfill\vfill\vfill
    
    \large
    \begin{tabular}[t]{ll}
        \textbf{Betreuer:}  & Prof. Guido Brunnett  \\
                            & M. Sc. Tom Uhlmann    \\
    \end{tabular}
}
\end{titlepage}

%\pagestyle{empty}

%-----------------------------
% Bibliographische Angaben
%-----------------------------
\thispagestyle{empty}
%\ \\[90ex]
%\null\vfill
%{
%    \ \\
%    \textbf{\dcauthorlastname, \dcauthorfirstname}\\
%    \dctitle\\
%    \dcsubject, Fakultät für Informatik\\
%    Technische Universität Chemnitz,~%
%        \ifcase\month%
%            \or Januar%
%            \or Februar%
%            \or März%
%            \or April%
%            \or Mai%
%            \or Juni%
%            \or Juli%
%            \or August%
%            \or September%
%            \or Oktober%
%            \or November%
%            \or Dezember%
%        \fi%
%    ~\number\year
%}
%\par\vfill\null

%-----------------------------
% Kurzzusammenfassung
%-----------------------------
%\clearpage
%\thispagestyle{empty}

%\null\vfill

\clearpage
\newpage

%\section*{\centering Zusammenfassung}

%Dieses Dokument gibt einen Überblick, wie eine wissenschaftliche Arbeit aufgebaut sein sollte. Dies gilt für Praktikumsberichte ebenso wie für Seminar-, Bachelor- und Masterarbeiten.

%\vfill
%\clearpage{\pagestyle{empty}\cleardoublepage}

%\pagestyle{headings}

%-----------------------------
% Verzeichnisse
%-----------------------------
\tableofcontents            \clearpage{\pagestyle{empty}\cleardoublepage}
%\listoffigures              \clearpage{\pagestyle{empty}\cleardoublepage}
%\listoftables               \clearpage{\pagestyle{empty}\cleardoublepage}
%\lstlistoflistings
%\clearpage
\thispagestyle{empty}
\chapter*{Ab\-kür\-zungs\-ver\-zeich\-nis\markboth{ABKÜRZUNGSVERZEICHNIS}{}}

\begin{acronym}[LaengereAbk.]
	\setlength{\itemsep}{-\parsep}\normalfont
	\acro{API}	{Application Programmers Interface}
	\acro{CIL}	{Common Intermediate Language}
	\acro{DBMS}	{Datenbankmanagementsystem}
	\acrodefplural{DBMS}{Datenbankmanagementsysteme}
	\acro{DML}	{Data Manipulation Language}
	\acro{DQL}	{Data Query Language}
	\acro{EDF}	{Earliest Deadline First}
	\acro{FIFO}	{First In First Out}
	\acro{FPGA}	{Field Programmable Gate Array}
	\acro{JIT}	{Just-in-time}
	\acro{LLVM}	{Low Level Virtual Machine}
	\acro{LLVM-IR}	{LLVM Intermediate Representation}
	\acro{RDBMS}	{Relationales Datenbankmanagementsystem}
	\acrodefplural{RDBMS}{Relationale Datenbankmanagementsysteme}
	\acro{RR}	{Round Robin}
	\acro{SQL}	{Structured Query Language}
	\acro{XML}	{Extensible Markup Language}
\end{acronym}
    \clearpage{\pagestyle{empty}\cleardoublepage}	
\mainmatter	% Zahlendarstellung auf Arabische Zahlen und von vorn anfangen
\acresetall

%---------------------------------------------------------------------------------------------------
%\chapter{Einleitung}
%% vim: set ts=4 sw=4 smartindent expandtab textwidth=100:

Die Einleitung dient der Vorstellung des Themas. Hier sollte für \emph{jeden} verständlich dargestellt werden, um was es geht. Es geht darum, eine Vorstellung davon zu vermitteln, was der Autor in dieser Arbeit geleistet hat.

Am Ende der Einleitung sollte ein grober Überblick über die Struktur der Arbeit gegeben werden.


\chapter{Aufgabenstellung}
% vim: set ts=4 sw=4 smartindent expandtab textwidth=100:

Im Sonderforschungsbereich Hybrid Societies wird die Interaktion von Menschen und
verkörperten digitalen Technologien (EDTs) untersucht. EDTs können autonome Fahrzeuge oder Roboter, Service- oder Info-Roboter, virtuelle Charaktere oder Menschen sein,
die mit technischen Erweiterungen wie intelligenten Brillen, Prothesen oder Exoskeletten
ausgestattet sind. Obwohl uns im täglichen Leben noch wenige dieser Dinge begegnen,
ergeben sich viele technische, logistische und soziale Implikationen, welche sich mit dem
Einzug solcher Technologien in unseren Alltag ergeben und bereits jetzt untersucht werden
sollten. Diese Implikationen betreffen unter anderem die Wahrnehmung, Kommunikation,
Interaktion, Umsetzung und den räumlichen Bedarf der jeweiligen Akteure. Die Erforschung dieser Aspekte ist schwierig, da die entsprechenden Geräte und Roboter entweder
noch nicht existieren oder sehr teuer sind. Daher ist es praktisch, eine virtuelle Umgebung
zu haben, in der diese Aspekte erforscht werden können.


In diesem teamorientierten Praktikum werden Möglichkeiten erforscht und umgesetzt,
wie eine lebendige, interaktive Welt erschaffen werden kann, in der ein Nutzer mit EDTs
interagieren kann oder die Interaktion von Agenten untereinander und in der Welt beobachten kann. Um dies zu erreichen, müssen die entsprechenden Objekte wie die Welt, die
Agenten und die Charaktere erschaffen werden. Zusätzlich müssen die Objekte interagierbar gemacht werden, und die Agenten müssen sich selbständig in der Welt bewegen und
mit ihr interagieren können. Der Fokus sollte dabei auf Glaubwürdigkeit und Plausibilität
liegen. Das bedeutet, dass Ansätze aus der wissenschaftlichen Literatur und bereits existierende EDTs verwendet werden sollten. Die virtuelle Welt sollte so angelegt sein, dass neue
Agenten in die Welt eingefügt oder das Verhalten bestehender Agenten geändert werden
kann, um die Auswirkungen zu untersuchen.

\chapter{Motivation}
% vim: set ts=4 sw=4 smartindent expandtab textwidth=100:

\section{Projektziel Innenstadt}

\begin{itemize}
\item modernes, menschenorientiertes, platzeffizientes Gelände
\begin{itemize}
\item modern, luxuriös -> geschwungene organische Formen, Steinoberfläche (Granit, Sandstein), Pflanzen, Glas
\item sehr vertikal, um platzeffizient zu sein, außerdem sieht das meist beeindruckend aus
\end{itemize}
\item Läden, in denen Roboter und Menschen arbeiten
\item belebte Umgebung, ohne dass es sich überfüllt anfühlt
\item Roboter machen Arbeit, die nicht erfüllend/stupide/schwer ist
\item Agenten:
\begin{itemize}
\item Putzroboter
\item Essensverkäufer
\item Gepäckroboter
\item Gärtnerroboter
\item Andere Menschen
\end{itemize}
\item Andere Systeme:
\begin{itemize}
\item AR-Brille
\item Informationstafeln/Hologramme
\end{itemize}
\end{itemize}

\section{Minimum Viable Product}

\begin{itemize}
\item Projektziel viel zu umfangreich
\item Minimieren auf ein realistischen Umfang, aber Grundidee soll erhalten bleiben
\item Szene sieht dann wie folgt aus:
\begin{itemize}
\item zwei Terassen mit Pflanzen
\item Rampe geht von einer Terasse zur nächsten
\item zwei Roboter gießen Pflanzen
\item Roboter finden Weg zu Pflanzen und vermeiden Kollision mit sich und Spieler
\item man kann Roboter fragen, was das für eine Pflanze ist
\end{itemize}
\end{itemize}

\chapter{Konzept}
% vim: set ts=4 sw=4 smartindent expandtab textwidth=100:

\section{Vorbedingungen}

Gemeinsam mit unserem Betreuer haben wir uns darauf geeinigt CrossForge zu verwenden. Die Engine unterstützt Windows, Linux und WebAssembly und benutzt nur OpenGL als Grafik API. Sie unterstützt physikalisch basiertes Rendering und Materialien, Deferred und Forward Rendering Pipelines und ein Shadersystem. Aber besonders wichtig für uns sind der Szenengraph, die Skelettanimationen und Text Rendering.

\section{Architektur und benötigte Systeme}

Aus dem MVP-Szenario und der Engine Wahl ergeben sich mehrere Anforderungen, die über verschiedene Systeme gelöst werden können. Die Roboter sind selbständige Agenten, die Entscheidungen treffen und diese auch ausführen sollen. Dieses Problem löst das Multiagentensystem. Sie müssen ihren Weg zur nächsten durstigen Pflanze finden, dafür ist das Navigationssystem zuständig. Die Roboter müssen die Pflanze gießen, was durch ein Partikelsystem visualisiert werden soll. Der Spieler und die Roboter sollen sich auf verschieden Plattformen bewegen und von einer zur nächsten laufen können. Zusätzlich sollen die Pflanzen verschiebbar sein. Um diese beiden Sachen zu realisieren, wird ein Physiksystem benötigt.
Der Spieler soll sich mit den Robotern unterhalten können, was durch das Dialogsystem abgedeckt wird.
Da CrossForge keinen eigenen Szenen Editor hat und die MVP Szene schon zu komplex ist, um diese mit Programmcode zu beschreiben, ist ein Szenen Editor nötig.

Um alle Systeme möglichst flexibel und unabhängig voneinander zu gestalten, haben wir uns für das Entity Component Pattern entschieden.

\section{Entity Component System}

\section{Partikelsystem}


\section{Multiagentensystem}

Das Multiagentensystem ist das Herz unseres Projektes da das Verhalten von allen belebten und unbelebten Agenten von diesem System gesteuert werden sollen. Man unterscheidet zwischen Zentraler KI und Agentenbasierter KI. Bei Zentraler KI werden die Entitäten von einem externen, globalen und allwissenden System gesteuert. Individuen haben deswegen keine Kontrolle über ihre eigenen Handlungen. Zentrale KI wird sehr oft eingesetzt, weil sich damit Gruppendynamiken und taktische Vorgehensweisen einfacher implementieren lassen. Bei Agentenbasierter KI treffen die Agenten unabhängige und individuelle Entscheidungen basierend auf den Informationen, die ihnen bereit stehen. Es gibt zwar trotzdem globale Informationen, die aber nicht dafür missbraucht werden dürfen die Handlungen eines Agenten zu diktieren. Wir haben uns für Agentenbasierte KI entschieden, weil sie für unser Szenario leichter umzusetzen ist. Die Hoffnung ist, dass die Akteure dadurch natürliche Verhaltensweisen und Interaktionen zur Schau stellen.

In verschiedenen GDC \cite{YouTube_2019}\cite{YouTube_2022}\cite{YouTube_2023} Talks wurde empfohlen, ein KI-Systen in mehreren Schichten aufzubauen:

\begin{itemize}
\item Sensoren
\item Entscheidungsfindung
\item Entscheidungsausführung
\item Bewegungssteuerung
\end{itemize}

\subsection{Sensoren}

Sensoren sind ein Querschnittskonzept und tauchen deshalb in allen Ebenen auf. Sie filtern Informationen aus der Umgebung und stellen diese der Schicht bereit, in der sie eingesetzt werden. Zusätzlich können Informationen über Blackboards mit anderen Agenten geteilt werden.

\subsection{Entscheidungsfindung}

Für die Entscheidungsfindung standen die folgenden Algorithmen in der näheren Auswahl: Beliefs, Desires, Intentions (BDI), Goal Oriented Action Planning (GOAP) und Finite State Machines (FSM). Am Ende haben wir uns für Finite State Machines entschieden, weil das das einfachste Verfahren war und vorerst für die Roboter ausreicht.

Eine State Machine ist ein gerichteter Graph mit limitierter Anzahl an Stati und Aktionen. Der Agent wechselt von einem Status in den nächsten, falls eine Bedingung erfüllt ist. In unserer Simulation sollen FSMs für die Roboter eingesetzt werden, um die Stati \textit{Pflanzen gießen}, \textit{Dialog mit Spieler}, \textit{Spieler folgen}, etc abdecken zu können. 

\subsection{Entscheidungsausführung}

Wenn der Agent eine Entscheidung getroffen hat, dann lässt sich diese Entscheidung meistens in weitere Teilprozesse zerlegen. Wenn ein Roboter zum Beispiel entschieden hat, dass er jetzt Pflanzen gießt, dann muss er:

\begin{itemize}
\item eine durstige Pflanze finden
\item zur Pflanze hin fahren
\item die Gießkanne zur Pflanze ausrichten
\item und schließlich die Pflanze gießen
\end{itemize}

Diese Sequenz von Handlungen lässt sich nur sehr schlecht mit FSMs darstellen, weswegen dieser Nachteil durch Behaviour Trees ausgeglichen werden soll. Behaviour Trees sind sehr gut darin solche Sequenzen darzustellen oder sogar nebenläufige Handlungen zu beschreiben. Ihr Nachteil ist jedoch, dass sie nur sehr schwierig auf Übergänge von einen Status in den nächsten reagieren können. Ein Beispiel hierfür ist, dass der Spieler den Roboter anspricht und ihn über Pflanzen befragt. Um das mit Behaviour Trees festzustellen, müssen überall Monitore eingebaut werden, die dieses Ereignis erkennen und behandlen. Das macht den Baum unübersichtlich und nicht wartbar. Behaviour Trees und State Machines ergänzen sich also sehr gut.

In Behaviour Trees werden Handlungen durch Knoten beschrieben. Die Knoten können die Stati: \textit{Laufend}, \textit{Fehler} oder \textit{Abgeschlossen} haben.
Knoten können dabei über die Bearbeitung ihrer Kindknoten entscheiden. Bei einem Sequenzknoten werden die Kinder von links nach rechts abgearbeitet. Nur wenn der Knoten den Status \textit{Abgeschlossen} hat, wird der rechte Geschwisterknoten aufgerufen. Wenn der Status \textit{Laufend} ist, dann wird der Knoten solange bearbeitet, bis der Status \textit{Abgeschlossen} oder \textit{Fehler} erreicht ist. Bei dem Status \textit{Fehler} wird die Abarbeitung abgebrochen und dieser nach oben propagiert. Der Elternknoten kann dann entscheiden, wie dieser Fehlerstatus behandelt wird.
Der Fallbackknoten behandelt den Fehler zum Beispiel, indem er den ersten Kindknoten findet, der nicht fehlschlägt und somit erfolgreich ausführt. Nur wenn alle Kindknoten fehlschlagen, ist der Status des Fallbackknotens \textit{Fehler}.

\subsection{Bewegungssteuerung}

Die Ebene der Bewegungssteuerung ist dafür verantwortlich die Entscheidungen in Beschleunigung, Geschwindigkeit und Rotation umzuwandeln. Dafür haben wir uns Steering Behaviour \cite{SteeringBehaviour} näher angeschaut. Es gibt verschiedene Bewegungsmuster:

\begin{itemize}
\item Seek
\item Wander
\item Collision Avoidance
\item Queue
\item ...
\end{itemize}

Für uns ist vor allem Collision Avoidance in Verbindung mit Seek interessant.

\section{Navigationssystem}

In der Mitte der MVP-Szene befindet sich die Wendeltreppe, weswegen die Roboter nicht immer den direkten Weg zur Pflanze fahren können, weil sie dieses Hindernis beachten müssen. Das Navigationssystem übernimmt die Aufgabe einen Pfad von einem Startpunkt zum Zielpunkt zu finden. Um Zeit zu sparen und uns auf die Hauptaufgabe konzentrieren zu können haben wir uns entschieden die Bibliothek Recast und Detour einzusetzen. Recast berechnet das Navigationsmesh aus der statischen Geometrie und Detour findet zur Laufzeit einen Pfad auf diesem Navigationsmesh. Da das Navigationsmesh nicht dynamisch angepasst werden muss, reicht es aus, wenn man es einmal vor der Kompilierung berechnet, speichert und dann in CrossForge lädt. Die Bibliothek stellt hierfür ein Beispielprogramm bereit, was wir nutzen.

\section{Physiksystem}

Die Entities und der Spieler sollen mit der statischen Szenengeometrie und sich selber kollidieren. Um diese Kollisionen zu erkennen und aufzulösen, ist das Physiksystem nötig. Dafür möchten wir die Bullet Bibliothek einsetzen, da Linus schon einmal den Separating Axis Theorem Algorihmus in 2D implementiert hat und das sich als sehr Zeitaufwändig und Fehleranfällig herausgestellt hat. 


\section{Szenen Editor}

Wir benötigen einen Szenen Editor, weil schon bei einer geringen Anzahl an platzierten Entitäten der Code unübersichtlich und schwer zu warten ist. Als alternative könnten wir die Agenten und Pflanzen prozedural platzieren, was aber ein zu komplexes Gebiet wäre und damit den Rahmen sprengen würde. Selber einen Szenen Editor von Grund auf neu zu schreiben ist keine Option, weil auch das eine riesige Aufgabe ist. Deswegen haben wir uns dazu entschieden die Open Source Software Blender zu verwenden. Blender ist ein mächtiger 3D-Editor den man über Addons erweitern kann. Somit haben wir zwei Möglichkeiten, unsere Szenen zu erstellen und in CrossForge zu laden. Wir können die Szene als GLTF-Datei exportieren und dann die vorhandenen Funktionen in CrossForge erweitern um aus der GLTF-Datei die einzelnen Transformationen für die Entitäten zu extrahieren. Oder wir schreiben ein Addon, welches die Szene nach unseren Anforderungen exportiert.

\subsection{GLTF als Szenenbeschreibung}

Khronos Group veröffentlichte am 19.10.2015 das offene GLTF Format, um dreidimensionale Szenen und Modelle darzustellen. In dem Format können Szenen mit ihren Knoten, Kamerainformationen, Animationen, Texturen, Materialien und natürlich auch Modellinformationen gespeichert werden. CrossForge unterstützt über die Assimp Bibliothek verschiedene Dateiformate, unter anderem auch GLTF.

Die Idee ist, dass alle Entities einer bestimmten Namenskonvention folgen. Die Szene wird als ganzes in das GLTF-Format exportiert und mit Assimp geladen. Assimp hat einen eigenen Szenengraph, den man traversieren kann und wenn man an einem Knoten mit bestimmer Namenskonvention angelangt ist, weiß man, dass es sich um ein Entity handelt. Die Transformation des Knotens wird dann zur Transformation des Entities. Alle Kindknoten können dann zu einem Modell zusammengefasst werden und für die Geometriekomponente des Entities benutzt werden.

Dieser Ansatz hat den Vorteil, dass er Editoragnostisch ist und wir können vorhandenes Wissen über Assimp nutzen. Leider hat diese Variante aber auch große Nachteile: man muss die Änderungen ziemlich tief in CrossForge vornehmen, was sehr viele Code Änderungen nach sich zieht. Der Szenenersteller muss die Knoten im Editor korrekt benennen, weil man sie in CrossForge sonst nicht mehr erkennen kann.

\subsection{Eigenes Format zur Szenenbeschreibung}

Wir können ein Blender Plugin schreiben, was die Transformationen der einzelnen Entities in eine JSON-Datei exportiert. Das Problem ist, dass Blender erkennen muss, was statische Geometrie ist und was Entities sind. Es gibt aber eine Funktion um externe .blend Dateien in die aktuelle zu linken. Jetzt kann man die Regel aufstellen, dass alles, was gelinkt ist ein eigenes Entity ist. Die Vorteile sind, dass keine Eingriffe ins Engine-Innere nötig sind. Da man das Dateiformat selber bestimmen kann, ist die Implementierung auf CrossForge Seite auch sehr simpel. Der Nachteil ist, dass noch niemand von uns ein Blender Addon geschrieben hat und wir deswegen noch keine Erfahrung in diesem Bereich haben.

\section{Dialogsystem}

\section{Modelle und Szene}

\chapter{Implementierung}
% vim: set ts=4 sw=4 smartindent expandtab textwidth=100:

\section{Entity Component System}

\section{Multiagenten System}

\subsection{Sate Machines}

nicht mehr dazu gekommen, weil Zeit vorbei war und Aufgaben von anderen Projektteilnehmern übernommen werden musste
aber auch nicht nötig, weil Dialogsystem noch nicht fertig war

\subsection{Behaviour Tree}

\subsection{Steering Behaviour}

\section{Partikelsystem}

\section{Navigationssystem}

Keine Besonderheiten. Ist halt so wie im Konzept

\section{Physiksystem}

Da das Physiksystem mit Observern feststellt, ob eine Physikkomponente hinzugefügt wurde, muss die Komponente schon initialisiert sein und man kann die lazy Initialisierung nicht benutzen. Deswegen muss man Physikkomponenten mit emplace anstatt add zum Entity hinzufügen.

Die Platform und Wendeltreppe ist statische Geometrie und zusätzlich noch Konkav. Deswegen haben wir diese mit einem Dreieckskollider ausgestattet. Dabei wird einfach das 3D-Modell als Collider benutzt.

Alle Entities haben Kapsel Collider, da die Entities sonst an den Dreieckskanten der statischen Geometrie hängen geblieben sind.

\section{Szenen Editor}

Die Assimp Variante hat nicht funktioniert, weil die eingriffe zu tief in der Engine vorgenommen werden mussten. Außerdem war das Abstraktionslevel zu gering: wenn man nur Knoten hat ist es schwer Entities zu erkennen.

Das Blender Plugin wurde in Python programmiert und man hat zugriff auf alle Optionen, auf die man auch im Editor Zugriff hat. Um jetzt die Entities zu exportieren, wurde über alle Objekte der Szene iteriert und überprüft, ob diese gelinkt sind. Wenn das der Fall ist, dann wurde die Transformation, der Name und der Name der Modelldatei in eine JSON-Datei geschrieben. Um jetzt auch noch die statische Geometrie zu exportieren, wurden alle Entities unsichtbar gemacht und nur die sichtbaren Modelle wurden exportiert.

In CrossForge wurde dann die JSON-Datei geladen und für jeden Eintrag wurde ein entsprechendes Entity zu Welt hinzugefügt.

\section{Dialogsystem}

Die gesamte Implementierung des Dialogsystems haben wir (Lisa und Sophie) im Pair-Coding übernommen. 

\subsection{Imgui}

Die Bibliothek für Imgui konnte über vcpkg recht schnell und einfach in das Projekt eingebunden werden. Zur Erzeugung von Dialogfenstern erstellen wir mit Imgui Frames, in denen oben der aktuelle Text des Dialogs angezeigt wird und unten die möglichen Antworten als Buttons. Wir nutzen außerdem einige Style-Anpassungen, um den Dialog etwas anschaulicher für den Nutzer zu gestalten, sodass z.B. das Fenster immer im Viewport zentriert ist.

\subsection{JsonCpp}

Ähnlich wie Imgui konnten wir auch JsonCpp über vcpkg einbinden. Mithilfe dieser Bibliothek können wir dann den Dialoggraphen initialisieren. Dafür sollten die Json-Dateien die gleiche Baumstruktur aufweisen, wie der Dialoggraph selbst. Über JsonCpp werden die Informationen aus der Datei gelesen und gespeichert, sodass sie dann weiter genutzt werden können, um den Dialoggraph zu füllen.

\subsection{Dialoggraph}

Den Dialoggraph haben wir so wie im Konzeptteil bechrieben implementiert. In einer dafür erstellten Klasse werden die zuvor genannten Daten gespeichert: der Dialogtext, ein Boolean für die Unterscheidung zwischen Roboter und Spieler und die möglichen Antworten. Zusätzlich gibt es noch eine Funktion, die die Initialisierung des Baumes umsetzt und eine Funktion für das Ersetzen der Strings mithlfe der Dialogmap.

\subsection{Dialogmap}

Mit der implementierten Dialogmap kann eine Funktion aufgerufen werden, die den Name einer Pflanze zurückgibt. Der Rückgabewert dieser Funktion ist allerdings statisch festgelegt, da zum Zeitpunkt der Abgabe das Dialogsystem noch nicht in das ECS eingebunden ist. Diese Integration wäre aber notwendig um die Positionen von Roboter und Spieler zu ermitteln, die gebraucht werden um zu bestimmen welcher Pflanzenname zurückgegeben werden soll. Auch die Pflanzennamen selbst sind im ECS gespeichert, weshalb ein Zugriff darauf vom Dialogsystem aus aktuell nicht möglich ist.  
Weitere Funktionalitäten der Dialogmap, wie Eventhandling sind ebenfalls noch nicht implementiert.

\section{Modelle und Szene}

Wie bereits im >Konzept< beschrieben ging die Anfangszeit in die Konzeptbearbeitung hinein. Ebenso hat es etwas Zeit in Anspruch genommen sich in Blender einzuarbeiten. Alle Modelle wurden mehrmals modelliert aufgrund von verschiedenen Problemen, wie z.B. Designunklarheiten, zu viele Polygone innerhalb des Models aufgrund zu vieler Details oder falscher Vorgehensweise, oder anderen Extras etc. >BILD EINFÜGEN ersten robomodels, ersten blätter mit wireframe<
Die Modelle wurden via Blender hergestellt und die jeweils dazugehörigen Texturen wurden selbstständig auf dem iPad mit ProCreate gezeichnet.
Der Roboter wurde grundsätzlich aus einfachen Zylindern modelliert und entsprechend angepasst. Die Grundüberlegung bei diesem war es, dass er nicht zu klassisch ‚standartrobotermäßig‘ aussieht, da man mit ihm kommunizieren kann. Die Solarplatte auf seinem Kopf wurde mit der Intension bzw. Überlegung integriert, dass er sich über die Sonne aufladen kann. Der Arm soll einen Schlauch darstellen, mit dem die Pflanzen gegossen werden können. Alle einzelnen Objekte, bis auf den Schlauch, wurden zum Schluss zusammen gemerged. Der Schlauch blieb als einzelnes Objekt, da er im späteren Verlauf animierbar sein soll. Was die Texturen angeht, wurden bis auf die Solarplatte, alle direkt in Blender via Geometry Nodes eingestellt. Daher bekam der Körper den metallischen, und die Räder den gummiartigen Look. Die Solarplatte wurde in ProCreate selbst gezeichnet und im UV-Editor eingefügt und angepasst. Nach Absprachen mit dem Team entstand dann unser Roboter wie man ihn aktuell im MVP sehen kann. >BILD EINFÜGEN< 
Die beiden Pflanzen sind relativ frei aus dem Kopf entstanden. Die einzigen Überlegungen dabei waren es, zwei unterschiedliche Pflanzen zu haben. Bei beiden entstanden die Blumentöpfe aus einfachen Zylindern. Der Stiel der kleineren Pflanze entstand anfangs erst aus einem einfachen Mesh mit verschiedenen Einstellungen der Geometry Nodes [1]. Diese Variante wurde allerdings verworfen und dann doch vereinfacht. Die Blätter, welche anfangs relativ 3D-getreu, aber zu detailliert waren bestehen jetzt nur noch aus zwei einfachen 2D-Texturen. Alle Texturen, sowohl Blätter als auch die Töpfe mit Erde, wurden in ProCreate selbstständig erstellt. Der erste versuch bestand daraus, die Modelle in ProCreate zu importieren und direkt darauf zu zeichnen. Diese Texturen hätten dann in Blender ganz einfach importiert werden können. Allerdings funktionierte diese Variante nicht ganz so wie erhofft, weswegen die Texturen dann einfach 2D gezeichnet wurden, und dann via UV-Mapping in Blender angepasst wurden. Wie auch bei dem Roboter wurden die Pflanzen nach Absprache mit dem Team abgesegnet und ins MVP aufgenommen. >BILD EINFÜGEN<
Die größere Hürde waren die Plattformen und deren Design. Wie bereits im Konzept beschrieben, entstanden da anfangs verschiedene Ideen. Beide Plattformen entstanden letzten Endes frei Hand und dazu eine etwas ausgefallenere Verbindung zwischen eben diesen Beiden. Damit die Roboter sich im Notfall auch zwischen den beiden Plattformen bewegen können, entstand dazu mit Inspiration noch eine Rampe um die Säule herum [2]. Damit sowohl die Rampe als auch die Plattformen etwas sicherer sind, bekamen diese noch jeweils einen Zaun und ein Geländer herum. Auch hier entstanden alle Texturen via ProCreate aus eigener Hand. Dieses Modell wurde ebenso mit dem Team abgesprochen, bevor es final war. >BILD EINFÜGEN< Am Ende wurde leider etwas zu spät aufgefallen, dass die Rampe breiter sein sollte, für den Fall, dass beide Roboter gleichzeitig die Ebenen wechseln wollen und sich auf der Rampe treffen. Zum aktuellen Zeitpunkt ist dort leider kein Platz zum Ausweichen und einer der Roboter driftet ab.
Nachdem alle Modelle fertig waren, wurden diese in der Szene zusammengesetzt und weitestgehend angepasst. Das Ganze wurde dann mithilfe eines Dropper Addon ins Projekt geladen. Hier gab es Anfangs einige Probleme und Unklarheiten, welche aber letzten Endes mit Hilfe des Teamleiters gelöst wurden. Um die Zusammenarbeit zwischen den Modellen und den restlichen Programmieraufgaben, wie das Navigationssystem, zu verbessern, wurden die Modelle in ‚linked‘ (Pflanzen und Roboter) und ‚static‘ (Plattformen & co) eingeteilt.
Als die Szene an sich so weit fertig war und auch im Projekt geladen war, gab es nur noch ein paar generelle Probleme zu lösen, was Texturen und einige Ordnerstrukturen anging.
Die nächste Aufgabe galt dann eigentlich die MVP-Szene weiter auszubauen und aufzuhübschen. Hierfür entstanden über ein neues Addon zwei Bäume und ein Haus. Auch hier wurden sowohl für die Bäume als auch für das Haus alle Texturen über ProCreate erstellt, wobei das Haus noch nicht ganz fertig war. >BILD EINFÜGEN< Aufgrund des Zeitlimits wurde der Ausbau der Szene allerdings vorzeitig abgebrochen.

\section{Management}

Der nächste Abschnitt ist aus der Ich-Perspektive des Projektmanagers (Linus) geschrieben, weil das einfacher ist.

Unser Praktikumsteam bestand aus sechs Personen mit unterschiedlichen Wissensständen, weswegen ich die Aufgaben generell nach Schwierigkeit und Fähigkeiten der Bearbeiter verteilt habe. Angelique bricht ihr Informatikstudium ab und bekommt deshalb die kreative Aufgabe die Modelle und Szene zu erstellen. Lisa und Sophie haben noch sehr wenig Erfahrung und sollten deshalb die einfachsten Themen angehen. Leon auch noch wenig Erfahrung aber dafür eine gewisse leidensfähigkeit. Mittelschwere bis schwere Aufgaben hat er ohne Widerstand in Angriff genommen. Carlo und ich haben die meiste Erfahrung und arbeiten auch schon mehrere Stunden in der Woche als Werkstudenten. Deswegen haben wir die uns auf schwersten Probleme fokusiert.

Als Team haben uns darauf geeinigt mindestens sechs Stunden in der Woche dem Teamorientierten Praktikum zu widmen, da mehr neben Studium und Arbeit nicht möglich war. Deswegen haben wir uns auch gegen einen Scrum Prozess entschieden, weil die Retros, Plannings und Sprintwechsel zu viel Zeit in Anspruch nehmen. Um uns zu organisieren wollten wir Jira einsetzen, jedoch hat die Software niemand wirklich benutzt. Da der Fortschritt langsam war konnte man auch ohne Hilfsmittel sehr gut den Überblick darüber behalten, wer an was gearbeitet hat.

Wir haben uns einmal in der Woche, entweder über Discord oder in Präsenz, getroffen um uns zum Projektfortschritt zu synchronisieren. Da die meisten mit ihrer Aufgabe überfordert waren und nicht wussten, wo sie anfangen sollten habe ich mit Lisa, Sophie und Leon jeweils eine Stunde in der Woche eine Pair-Coding-Session gehalten. Das hatte den großen Vorteil, dass ich dann genau wusste, wer welchen Fortschritt hat. Dem gegenüber standen zwei große Nachteile. Erstens konnte ich somit keine Features implementieren, weil meine sechs Stunden voll mit Pair Coding und Management waren und zweitens ist es sehr verlockend bei jeder kleinen Hürde aufzugeben und auf das nächste Pair Coding zu warten, weil es dort ja sowieso gelöst wird.

Nach den Sommerferien habe ich eine Retrospektive vorbereitet und mit dem gesamten Team durchgeführt. Dabei haben sich verschiedene bevorzugte Arbeitsweisen heraus kristallisiert. Während Leon die Pair Coding Sessions bevorzugt und weiter führen möchte, sagen Lisa und Sophie, dass sie ihnen wenig bringen und eigenständig weiter arbeiten möchten. Auf diese Art und Weise haben wir dann auch bis zum Ende weiter gearbeitet.

Projektmanagement ist als Student schwer, weil man auf die intrinsische Motivation der Teammitglieder hoffen muss. Manager in der Industrie haben richtige Druckmittel (Mitarbeiter feuern) oder extrinische Motivation (Lohn). Beides kann ich als studentischer Leiter nicht einsetzen. Zusätzlich ist es auch so, dass harte Arbeit und Engagement mit noch mehr Arbeit bestraft wird, weil sich diese Person als verlässlich herausgestellt hat. Es lohnt sich für Gruppenarbeiten also mehr, wenn man nichts macht, weil man am Ende \"doch immer besteht\".

Das Dialogsystem, an dem Lisa und Sophie gearbeitet haben, wurde nicht rechtzeitig fertig weil es noch in das ECS integriert werden musste und das eine sehr zeitaufwendige Aufgabe ist. Ich wurde damit beauftragt einen Abschnitt zu schreiben, wie das passiert ist und deswegen habe ich mir die Git-Historie näher angeschaut. Generell habe ich darauf geachtet, dass sich die einzelnen Feature Branches nicht zu weit voneinander entfernen. Deswegen habe ich am 09.10.2023 alle Branches in den Master gemerged und danach den Master wieder in alle Featurebranches. Damit war der komplette Projektstand synchronisiert und die beiden hätten anfangen können das Dialogsystem in das Entity Component System zu integrieren. Wir haben aber gemeinsam entschieden, dass wir noch Zeit haben und das Dialogsystem zuerst fertig gestellt werden soll. Die beiden haben dann am 05.12.2023 begonnen das Dialogsystem in das Entity Component System zu integrieren. Rückblickend wäre es sinnvoller gewesen zuerst die Integration zu machen und danach das Dialogsystem fertig zu stellen.

Carlo hat das Praktikum vor der Bewertung abgebrochen, weil er nicht genügend Zeit hatte, um einen sinnvollen Beitrag zu leisten. Er war der mit der meisten C++ Erfahrung und arbeitet 18 Stunden in der Woche als Werkstudent, weswegen er die Aufgabe mit der größten unsicherheit an Komplexität lösen sollte. Seine Aufgabe war das Pathfinding System in CrossForge zu integrieren. Aufgrund von Studium, Arbeit, Umzug, Problemen mit seiner Entwicklungsumgebung und mäßiger Dokumentation von Detour hat er nur sehr langsam Fortschritt gemacht.  Bei der Retrospektive haben wir über diese Probleme gesprochen und ich habe ihm vorgeschlagen eine andere, leichtere Aufgabe zu übernehmen. Er wollte nochmal einen Anlauf wagen hat aber dann nach zwei Wochen mein Angebot angenommen. Ich habe seine Aufgabe übernommen und er sollte als nächstes das Partikelsystem implementieren. Durch seine kontinuierlichen IDE Probleme hat er aber die Lust am programmieren verloren. Das Partikelsystem hat dann Leon implementiert und Carlo sollte ab 12.12.2023 Angelique bei den Modellen und einer hübscheren Szene helfen. Er hat in der Woche vor Weihnachten noch die bestehende Szene ein wenig verbessert. Dann waren Weihnachtsferien und im Januar haben wir uns voll dem Bericht schreiben gewidmet. Somit hatte Carlo keine Chance mehr, genug zum Projekt beizutragen, um das Praktikum zu bestehen.


%\addtocontents{toc}{\protect\pagebreak[4]} % Seitenumbruch im Inhaltsverzeichnis

\chapter{Ergebnisse}
% vim: set ts=4 sw=4 smartindent expandtab textwidth=100:

Jede Arbeit hat normalerweise Ergebnisse. Dies können Messreihen, Beweise und vieles mehr sein. In diesem Kapitel werden die Ergebnisse präsentiert und diskutiert. Meist ist die Implementation nicht vollkommen und zeigt in Randbereichen Schwächen. Hier ist der Platz dies aufzuzeigen.


\chapter{Zusammenfassung \& Ausblick}
% vim: set ts=4 sw=4 smartindent expandtab textwidth=100:
In diesem Teamorientierten Praktikum haben wir grundlegende Systeme geschaffen, um Multiagentensimulationen durchzuführen. 

Als Grundlage haben wir CrossForge genutzt und in diese Engine ein Pathfinding System, einen flexiblen KI-Stack und eine Physikengine integriert. Zusätzlich haben wir einen Szeneneditor für CrossForge geschrieben und die Grundsteine für ein Dialogsystem gelegt.

Das Zusammenspiel der Komponenten haben wir an einer Beispielszene gezeigt, in der Roboter sich durch die Welt bewegen und Pflanzen gießen.


Als nächstes würden wir das Dialogsystem vervollständigen und Animationen zu den Akteuren hinzufügen. Danach würden wir verschiedene Agenten implementieren und die Szene erweitern.

\begin{appendix}

\chapter{Nachsätze}
% vim: set ts=4 sw=4 smartindent expandtab textwidth=100:


\section{Arbeitsaufteilung}

\begin{itemize}
\item Angelique: Modelle
\item Lisa und Sophie: Dialogsystem
\item Leon:
\begin{itemize}
\item Anbindung von Entity Component System: Flecs
\item Implementierung vom Multiagentensystem
\item Konzept und Implementierung Partikelsystem
\end{itemize}
\item Linus:
\begin{itemize}
\item Projekt Management
\item Architektur
\item Konzept Multiagentensystem
\item Konzept und Implementierung von Navigationssystem, Physiksystem, Szenen Editor
\end{itemize}
\end{itemize}

\begin{figure}[h!]
\begin{tabular}{|c|c|c|c|c|c|c|}
\hline
Metric        & Leon         & Sophie    & Carlo     & Lisa      & Linus         & Angelique \\
\hline
Insertions    & 1.877 (7\%)  & 341 (1\%) & 296 (1\%) & 140 (0\%) & 23.092 (82\%) & 2.305 (8\%) \\
\hline
Deletions     & 1.010 (20\%) & 143 (3\%) & 6 (0\%)   & 43 (1\%)  & 3.775 (76\%)  & 3 (0\%) \\
\hline
Files         & 141 (35\%)   & 16 (4\%)  & 2 (0\%)   & 6 (1\%)   & 230 (57\%)    & 7 (2\%) \\
\hline
Commits       & 39 (33\%)    & 6 (5\%)   & 3 (3\%)   & 2 (2\%)   & 64 (55\%)     & 3 (3\%) \\
\hline
Lines changed & 2.887 (9\%)  & 484 (1\%) & 302 (1\%) & 183 (1\%) & 26.869 (81\%) & 2.308 (7\%) \\
\hline
\end{tabular}
\caption{Arbeitsverteilung anhand Git Historie im Zeitraum von 01.06.2023 bis 22.01.2024 \cite{gitquickstats}}
\label{arbeitsaufteilung}
\end{figure}

\end{appendix}

%---------------------------------------------------------------------------------------------------
\backmatter

%---------------------------------------------------------------------------------------------------
\printindex

%---------------------------------------------------------------------------------------------------
%Literatur, die nicht direkt referenziert wird mit \nocite auffuehren
\addcontentsline{toc}{chapter}{Literaturverzeichnis}
\begin{sloppypar}
	\printbibliography
\end{sloppypar}

%---------------------------------------------------------------------------------------------------
\clearpage
\thispagestyle{empty}
%\addcontentsline{toc}{chapter}{Selbständigkeitserklärung}

%\newgeometry{left=2cm,right=3.5cm,top=2cm,bottom=2cm}

%\newlength\us
%\settowidth{\us}{-\dcauthorfirstname~\dcauthorlastname-}

%\begin{overpic}[width=\linewidth,grid, tics=1]{pics/selbstaendigkeitserklaerung}
%\begin{overpic}[width=\linewidth]{pics/selbstaendigkeitserklaerung}
%    \put (6.5,81.71) {\small\sffamily \dcauthorlastname}
%    \put (6.5,78.15) {\small\sffamily \dcauthorfirstname}
%    \put (6.5,74.60) {\small\sffamily \dcauthorbirthdate}
%    \put (6.5,71.05) {\small\sffamily \dcmatriculation}
%    \put (5,32.69)   {\small\sffamily \today}
%    \put (45.8,57.46){\small\sffamily \dcsubject}
%    \put (38,32.6)   {\rule[-0.1ex]{\us}{0.5pt}}
%\end{overpic}
%\restoregeometry

\end{document}
