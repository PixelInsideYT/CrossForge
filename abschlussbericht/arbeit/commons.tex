% Packages, die ueberall eingebunden werden muessen, z.B. damit Definitionen in dieser
% Datei funktionieren

%Default TU- und Fakultaetslogo
\newcommand{\tulogo}{pics/TU_Chemnitz_positiv_gruen}

% Hier den Namen des Verfassers eintragen, das ist eine Variable, weil sie eventuell oefters
% im Dokument vorkommt.
\newcommand{\dctitle}		{Abschlussbericht}
\newcommand{\dcauthortitle}	{Herr B.\,Sc.}
\newcommand{\dcauthorbirthdate}	{1. April 2000}
\newcommand{\dcauthorbirthplace}{Chemnitz}
\newcommand{\dcauthorlastname}	{Mustermann}
\newcommand{\dcauthorfirstname}	{Max}
\newcommand{\dcsubject}		{Masterarbeit}
\newcommand{\dcmatriculation}	{123456}

% ein kleines Hilfsmakro, um TODOs zu markieren
%\newcommand{\todo}[1]		{\hspace{0cm}\marginpar{\includegraphics[]{pics/todo.eps}}\textbf{-- TODO: #1 --}}
\newcommand{\todo}[1]		{}

% ein kleines Hilfmakro, mit dem ein Text am Anfang eines Gliederungspunktes
% eingefuegt werden soll, in dem ausgesagt wird, was gesagt werden soll.
% Fuer die finale Version wird dieses Makro einfach leer definiert.
%\newcommand{\whatsthis}[1]	{\hspace{0cm}\marginpar{\includegraphics[]{pics/what.eps}}{--- #1 ---}}
\newcommand{\whatsthis}[1]	{}
